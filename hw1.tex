\documentclass[12pt]{article}		% The percent symbol in your code starts a comment.  The comment ends at the next linebreak.

\usepackage[english]{babel} 		% Packages add functionality and style conventions to your documents. Don't edit this section!
\usepackage{fullpage}				% Eliminates wasted space
\usepackage[utf8]{inputenc}			% Necessary for character encoding
\usepackage{amsmath, amssymb,amsthm}% Required math packages
\usepackage{graphicx}				% For handling graphics
\usepackage[colorinlistoftodos]{todonotes}	% For the fancy "todo" stuff
\usepackage{hyperref}				% For clickable links in the final PDF
\usepackage{titling}				% To take less space at the top of the page with the title
\setlength{\droptitle}{-2cm}
\pretitle{\begin{flushright}\Large\scshape}
\posttitle{\par\end{flushright}}
\preauthor{\begin{flushright}\large\scshape}
\postauthor{\par\end{flushright}}
\predate{\begin{flushright}\large\scshape}
\postdate{\par\end{flushright}}


\setcounter{section}{1}

\theoremstyle{definition}
\newtheorem{theorem}{Theorem}[section]
\newtheorem{lemma}[theorem]{Lemma}
\newtheorem{prop}[theorem]{Proposition}
\newtheorem{claim}[theorem]{Claim}
\newtheorem{example}[theorem]{Example}
\newtheorem{exercise}[theorem]{Exercise}

\newtheorem*{ex}{Exercise}

%  Here is an easy way to make lines, segments, and rays just type $ \lin{AB}$ for example.
\newcommand{\lin}[1]{\overleftrightarrow{#1}}
\newcommand{\seg}[1]{\overline{#1}}
\newcommand{\ray}[1]{\overrightarrow{#1}}


\title{Math 208 HW 1}

\author{Your name goes here}


\date{Due September 1, 2020}

\begin{document}
	\maketitle
	
	
	\begin{ex}[3.2.17 (Existence and Uniqueness of Midpoints) ]
		Prove the following theorem.  If $A$ and $B$ are distinct points, there there exists a unique point $M$ such that $M$ is the midpoint of $\seg{AB}$.
	\end{ex}
	
	\begin{proof} 
		% % % % Type your proof here!  
		
	\end{proof}

\vspace{1in} % leave some space in here so I can write comments	
	
	\vspace{.25in}
	
	
	\begin{ex}[3.2.19 (Segment Addition Theorem) ]
		Prove the following theorem.  If $A * B * C$, $D * E *F$, $\seg{AB} \cong \seg{DE}$, and $\seg{BC} \cong \seg{EF}$, then $\seg{AC} \cong \seg{DF}$.
	\end{ex}
	
	\begin{proof} 
		% % % % Type your proof here!  
		
	\end{proof}


\vspace{1in} % leave some space in here so I can write comments	
	
	
	
\end{document}
