\documentclass[12pt]{article}		% The percent symbol in your code starts a comment.  The comment ends at the next linebreak.

\usepackage[english]{babel} 		% Packages add functionality and style conventions to your documents. Don't edit this section!
\usepackage{fullpage}				% Eliminates wasted space
\usepackage[utf8]{inputenc}			% Necessary for character encoding
\usepackage{amsmath, amssymb,amsthm}% Required math packages
\usepackage{graphicx}				% For handling graphics
\usepackage[colorinlistoftodos]{todonotes}	% For the fancy "todo" stuff
\usepackage{hyperref}				% For clickable links in the final PDF
\usepackage{titling}				% To take less space at the top of the page with the title

\setlength{\droptitle}{-2cm}

\setcounter{section}{1}

\theoremstyle{definition}
\newtheorem{theorem}{Theorem}[section]
\newtheorem{lemma}[theorem]{Lemma}
\newtheorem{prop}[theorem]{Proposition}
\newtheorem{claim}[theorem]{Claim}
\newtheorem{example}[theorem]{Example}
\newtheorem{exercise}[theorem]{Exercise}

\newtheorem*{ex}{Exercise}

%  Here is an easy way to make lines, segments, and rays just type $ \lin{AB}$ for example.
\newcommand{\lin}[1]{\overleftrightarrow{#1}}
\newcommand{\seg}[1]{\overline{#1}}
\newcommand{\ray}[1]{\overrightarrow{#1}}


\title{Math 208 HW 5}

\author{Your name goes here}


\date{Due October 9}

\begin{document}
	\maketitle
	
	


\begin{ex}[4.4.1]  Prove the Corresponding Angles Theorem (Corollary 4.4.4): If $\ell$ and $\ell'$ are lines cut by a transversal $t$ in such a way that two corresponding angles are congruent, the $\ell$ is parallel to $\ell'$.
	
\end{ex}

\begin{proof} 
	% % % % Type your proof here!  
	
\end{proof}



\vspace{1in} % leave some space in here so I can write comments



\begin{ex}[4.5.1] Prove Corollary 4.5.6.  The sum of the measures of two interior angles of a triangle is less than or equal to the measure of their remote exterior angle.
	
\end{ex}

\begin{proof} 
	% % % % Type your proof here!  
	
\end{proof}



\vspace{1in} % leave some space in here so I can write comments




	
	
	
\end{document}
