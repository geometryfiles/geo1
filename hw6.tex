\documentclass[12pt]{article}		% The percent symbol in your code starts a comment.  The comment ends at the next linebreak.

\usepackage[english]{babel} 		% Packages add functionality and style conventions to your documents. Don't edit this section!
\usepackage{fullpage}				% Eliminates wasted space
\usepackage[utf8]{inputenc}			% Necessary for character encoding
\usepackage{amsmath, amssymb,amsthm}% Required math packages
\usepackage{graphicx}				% For handling graphics
\usepackage[colorinlistoftodos]{todonotes}	% For the fancy "todo" stuff
\usepackage{hyperref}				% For clickable links in the final PDF
\usepackage{titling}				% To take less space at the top of the page with the title

\pretitle{\begin{flushright}\Large\scshape}
\posttitle{\par\end{flushright}}
\preauthor{\begin{flushright}\large\scshape}
\postauthor{\par\end{flushright}}
\predate{\begin{flushright}\large\scshape}
\postdate{\par\end{flushright}}

\setlength{\droptitle}{-2cm}

\setcounter{section}{1}

\theoremstyle{definition}
\newtheorem{theorem}{Theorem}[section]
\newtheorem{lemma}[theorem]{Lemma}
\newtheorem{prop}[theorem]{Proposition}
\newtheorem{claim}[theorem]{Claim}
\newtheorem{example}[theorem]{Example}
\newtheorem{exercise}[theorem]{Exercise}

\newtheorem*{ex}{Exercise}

%  Here is an easy way to make lines, segments, and rays just type $ \lin{AB}$ for example.
\newcommand{\lin}[1]{\overleftrightarrow{#1}}
\newcommand{\seg}[1]{\overline{#1}}
\newcommand{\ray}[1]{\overrightarrow{#1}}


\title{Math 208 HW 6}

\author{Your name goes here}


\date{Due October 13, 2020}

\begin{document}
	\maketitle
	
	\begin{ex}[4.7.1]
	Prove that Euclid's Fifth Postulate implies the Euclidean Parallel Postulate (see Theorem 4.7.2), namely assume that  ``If $\ell$ and $\ell'
	$ are parallel lines and $t$ is a transversal in such a way that the sum of the measures of the two interior angles on one side of $t$ is less than $180^\circ$, then $\ell$ and $\ell'$ intersect on that side of $t$.  Then you should prove that given any line $\ell$ and a point $P$ not on $\ell$ there is exactly one line through $P$ that is parallel to $\ell$.
\end{ex}

\begin{proof} 
	% % % % Type your proof here!
	
\end{proof}

\vspace{1in} % leave some space in here so I can write comments





\begin{ex}[4.7.6]
	Prove that the Euclidean Parallel Postulate implies that the angle sum of any triangle is $180^\circ$.  (The second half of Theorem 4.7.4).
\end{ex}

\begin{proof} 
	% % % % Type your proof here!
	
\end{proof}


\vspace{1in} % leave some space in here so I can write comments





\begin{ex}[4.8.5]
	Prove that every Saccheri quadrilateral has the properties listed in Theorem 4.8.10.  That is, if $\square ABCD$ is a Saccheri quadrilateral with base $\seg{AB}$, then
	\begin{enumerate}
		\item the diagonals $\seg{AC}$ and $\seg{BD}$ are congruent,
		\item the summit angles $\angle BCD$ and $\angle ADC$ are congruent,
		\item the segment jointing the midpoint of $\seg{AB}$ to the midpoint of $\seg{CD}$ is perpendicular to both $\seg{AB}$ and $\seg{CD}$,
		\item $\square ABCD$ is a parallelogram,
		\item $\square ABCD$ is a convex quadrilateral,
		\item the summit angles $\angle BCD$ and $\angle ADC$ are either right or acute.
	\end{enumerate}
\end{ex}

\begin{proof} 
	% % % % Type your proof here!
	
\end{proof}

\vspace{1in} % leave some space in here so I can write comments




\begin{ex}[4.8.6]
	Prove the following: If $\ell$ and $m$ are two distinct lines and there exist distinct points $P$ and $Q$ on $m$ such that $d(P,\ell) = d(Q,\ell)$, then either $m$ and $\ell$ intersect at the midpoint on $\seg{PQ}$, or $m \parallel \ell$.
\end{ex}



\vspace{1in} % leave some space in here so I can write comments


	
	
	
\end{document}
