\documentclass[11pt]{article}		% The percent symbol in your code starts a comment.  The comment ends at the next linebreak.

\usepackage[english]{babel} 		% Packages add functionality and style conventions to your documents. Don't edit this section!
\usepackage{fullpage}				% Eliminates wasted space
\usepackage[utf8]{inputenc}			% Necessary for character encoding
\usepackage{amsmath, amssymb,amsthm}% Required math packages
\usepackage{graphicx}				% For handling graphics
\usepackage[colorinlistoftodos]{todonotes}	% For the fancy "todo" stuff
\usepackage{hyperref}				% For clickable links in the final PDF
\usepackage{titling}				% To take less space at the top of the page with the title
\setlength{\droptitle}{-2cm}
\pretitle{\begin{flushright}\Large\scshape}
\posttitle{\par\end{flushright}}
\preauthor{\begin{flushright}\large\scshape}
\postauthor{\par\end{flushright}}
\predate{\begin{flushright}\large\scshape}
\postdate{\par\end{flushright}}


% Type `\R' for the real numbers, `\Z' for the integers, etc.

\def\Z{{\mathbb Z}}
\def\Q{{\mathbb Q}}
\def\R{{\mathbb R}}
\def\N{{\mathbb N}}

%  Here is an easy way to make lines, segments, and rays just type $ \lin{AB}$ for example.
\newcommand{\lin}[1]{\overleftrightarrow{#1}}
\newcommand{\seg}[1]{\overline{#1}}
\newcommand{\ray}[1]{\overrightarrow{#1}}

\setcounter{section}{1}

\theoremstyle{definition}
\newtheorem{theorem}{Theorem}[section]
\newtheorem{lemma}[theorem]{Lemma}
\newtheorem{prop}[theorem]{Proposition}
\newtheorem{claim}[theorem]{Claim}
\newtheorem{example}[theorem]{Example}
\newtheorem{exercise}[theorem]{Exercise}

\newtheorem*{ex}{Exercise}



\title{Math 208 HW 10}


\author{Your name goes here}  %%% Don't forget to put your name here!  %%%%


\date{Due November 10, 2020}

\begin{document}
	\maketitle
	
	%%%%%%%%%   Don't forget to use \newpage to start a new page to make 
	%%%%%%%%%   proofs/diagrams fit on the same page. 
	

\begin{ex}[6.6.1]
Prove Theorem 6.6.3, Part 1.  Namely, prove the following:  Suppose $\ell \parallel m$. Let $P$, $Q$, and $R$ be points on $m$ such that $P * Q * R$ and let $A$, $B$, and $C$ be the feet of the perpendiculars from $P$, $Q$, and $R$ to $\ell$. If $\lin{PA} \perp m$, then $PA < QB < RC$.
	
\end{ex}

\begin{proof} 
	% % % % Type your proof here!
	
\end{proof}


\vspace{1in} % leave some space in here so I can write comments





\begin{ex}[6.6.2]
	Prove Theorem 6.6.3, Part 2.  Namely, prove the following:  Suppose $\ell \parallel m$. Let $P$, $Q$, and $R$ be points on $m$ such that $P * Q * R$ and let $A$, $B$, and $C$ be the feet of the perpendiculars from $P$, $Q$, and $R$ to $\ell$. If $\ray{PQ} | \ray{AB}$, then $PA > QB > RC$.
	
\end{ex}

\begin{proof} 
	% % % % Type your proof here!
	
\end{proof}


\vspace{1in} % leave some space in here so I can write comments







\end{document}
