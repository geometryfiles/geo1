\documentclass[11pt]{article}		% The percent symbol in your code starts a comment.  The comment ends at the next linebreak.

\usepackage[english]{babel} 		% Packages add functionality and style conventions to your documents. Don't edit this section!
\usepackage{fullpage}				% Eliminates wasted space
\usepackage[utf8]{inputenc}			% Necessary for character encoding
\usepackage{amsmath, amssymb,amsthm}% Required math packages
\usepackage{graphicx}				% For handling graphics
\usepackage[colorinlistoftodos]{todonotes}	% For the fancy "todo" stuff
\usepackage{hyperref}				% For clickable links in the final PDF
\usepackage{titling}				% To take less space at the top of the page with the title
\setlength{\droptitle}{-2cm}
\pretitle{\begin{flushright}\Large\scshape}
\posttitle{\par\end{flushright}}
\preauthor{\begin{flushright}\large\scshape}
\postauthor{\par\end{flushright}}
\predate{\begin{flushright}\large\scshape}
\postdate{\par\end{flushright}}


% Type `\R' for the real numbers, `\Z' for the integers, etc.

\def\Z{{\mathbb Z}}
\def\Q{{\mathbb Q}}
\def\R{{\mathbb R}}
\def\N{{\mathbb N}}

%  Here is an easy way to make lines, segments, and rays just type $ \lin{AB}$ for example.
\newcommand{\lin}[1]{\overleftrightarrow{#1}}
\newcommand{\seg}[1]{\overline{#1}}
\newcommand{\ray}[1]{\overrightarrow{#1}}

\setcounter{section}{1}

\theoremstyle{definition}
\newtheorem{theorem}{Theorem}[section]
\newtheorem{lemma}[theorem]{Lemma}
\newtheorem{prop}[theorem]{Proposition}
\newtheorem{claim}[theorem]{Claim}
\newtheorem{example}[theorem]{Example}
\newtheorem{exercise}[theorem]{Exercise}

\newtheorem*{ex}{Exercise}


\title{Math 208 HW 7}

\author{Your name goes here}


\date{Due October 20, 2020}

\begin{document}
	\maketitle
	
	%%%%%%%%%   Don't forget to use \newpage to start a new page to make 
	%%%%%%%%%   proofs/diagrams fit on the same page. 
	
	


	
	\begin{ex}[5.1.1]
		State and prove the Converse to the Corresponding Angles Theorem.
		
	\end{ex}
	
	\begin{proof} 
		% % % % Type your proof here!
		
	\end{proof}
	
	
\vspace{1in} % leave some space in here so I can write comments
	
	
	
	
	\begin{ex}[5.1.3]
		Suppose $\square ABCD$ is a quadrilateral such that $\lin{AB} \parallel \lin{CD}$ and $\seg{AB} \cong \seg{CD}$.  Prove that $\square ABCD$ is a parallelogram.
		
	\end{ex}
	
	\begin{proof} 
		% % % % Type your proof here!
		
	\end{proof}
	
	
	
\vspace{1in} % leave some space in here so I can write comments
	


	
\begin{ex}[5.1.5]
	Properties of 60-60-60 and 30-60-90 triangles. An \textit{equilateral} triangle is one in which all three sides have equal lengths.
	
	\begin{enumerate}
		\item[(a)] Prove that a Euclidean triangle is equilateral if and only if each of its angles measures $60^\circ$.
		\item[(b)] Prove that there is an equilateral triangle in Euclidean geometry.
		\item[(c)] Split an equilateral triangle at the midpoint of one side to prove that there is a triangle whose angles measure $30^\circ$, $60^\circ$, and $90^\circ$.
		\item[(d)] Prove that, in any 30-60-90 triangle, the length of the side opposite the $30^\circ$ angle is one half the length of the hypotenuse.
	\end{enumerate}
	
\end{ex}

\begin{proof}  $\,$
	% % % % Type your proof here!
	
	\begin{enumerate}
		\item[(a)]
		\item[(b)]
		\item[(c)]
		\item[(d)]
	\end{enumerate}
	
\end{proof}



\vspace{1in} % leave some space in here so I can write comments







\end{document}
