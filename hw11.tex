\documentclass[11pt]{article}		% The percent symbol in your code starts a comment.  The comment ends at the next linebreak.

\usepackage[english]{babel} 		% Packages add functionality and style conventions to your documents. Don't edit this section!
\usepackage{fullpage}				% Eliminates wasted space
\usepackage[utf8]{inputenc}			% Necessary for character encoding
\usepackage{amsmath, amssymb,amsthm}% Required math packages
\usepackage{graphicx}				% For handling graphics
\usepackage[colorinlistoftodos]{todonotes}	% For the fancy "todo" stuff
\usepackage{hyperref}				% For clickable links in the final PDF
\usepackage{titling}				% To take less space at the top of the page with the title
\setlength{\droptitle}{-2cm}
\pretitle{\begin{flushright}\Large\scshape}
\posttitle{\par\end{flushright}}
\preauthor{\begin{flushright}\large\scshape}
\postauthor{\par\end{flushright}}
\predate{\begin{flushright}\large\scshape}
\postdate{\par\end{flushright}}


% Type `\R' for the real numbers, `\Z' for the integers, etc.

\def\Z{{\mathbb Z}}
\def\Q{{\mathbb Q}}
\def\R{{\mathbb R}}
\def\N{{\mathbb N}}

%  Here is an easy way to make lines, segments, and rays just type $ \lin{AB}$ for example.
\newcommand{\lin}[1]{\overleftrightarrow{#1}}
\newcommand{\seg}[1]{\overline{#1}}
\newcommand{\ray}[1]{\overrightarrow{#1}}

\setcounter{section}{1}

\theoremstyle{definition}
\newtheorem{theorem}{Theorem}[section]
\newtheorem{lemma}[theorem]{Lemma}
\newtheorem{prop}[theorem]{Proposition}
\newtheorem{claim}[theorem]{Claim}
\newtheorem{example}[theorem]{Example}
\newtheorem{exercise}[theorem]{Exercise}

\newtheorem*{ex}{Exercise}



\title{Math 208 HW 11}


\author{Your name goes here}  %%% Don't forget to put your name here!  %%%%


\date{Due November 24, 2020}

\begin{document}
	\maketitle
	
	%%%%%%%%%   Don't forget to use \newpage to start a new page to make 
	%%%%%%%%%   proofs/diagrams fit on the same page. 
	

\begin{ex}[8.1.2]
Prove that the points on a tangent line lie outside the circle  (Theorem 8.1.8).
	
\end{ex}

\begin{proof} 
	% % % % Type your proof here!
	
\end{proof}


\vspace{1in} % leave some space in here so I can write comments




\begin{ex}[8.1.5]
	Prove the Tangent Circles Theorem (Theorem 8.1.15).
	
\end{ex}

\begin{proof} 
	% % % % Type your proof here!
	
\end{proof}


\vspace{1in} % leave some space in here so I can write comments





\begin{ex}[10.1.2]
	Let $S$ be a set of points in the plane.  A line $\ell$ is called a \textbf{line of symmetry} for $S$ if $\rho_\ell(S) = S$.  Find all lines of symmetry for each of the following.  Explain why, and illustrate with a diagram.
	\begin{enumerate}
		\item[(a)] An isosceles triangle.
		\item[(b)] An equilateral triangle.
		\item[(c)] A regular polygon.
		\item[(d)] A circle.
	\end{enumerate}
	\vspace{.1in}
\end{ex}

\begin{proof}
	$\,$
	% % % % Type your proof here!
	
	\begin{enumerate}
		\item[(a)] An isosceles triangle.

		\item[(b)] An equilateral triangle.
		
		\item[(c)] A regular polygon.
		
		\item[(d)] A circle.
	\end{enumerate}
	
\end{proof}


\vspace{1in} % leave some space in here so I can write comments





\begin{ex}[10.1.9]
	Prove that an isometry preserves circles. (Part 8 of Theorem 10.1.7)  That is prove that if $T$ is an isometry and $\gamma$ is a circle with center $O$ and radius $r$, then $T(\gamma)$ is a circle with center $T(O)$ and radius $r$.
	
\end{ex}

\begin{proof} 
	% % % % Type your proof here!
	
\end{proof}





\end{document}
