\documentclass[11pt]{article}		% The percent symbol in your code starts a comment.  The comment ends at the next linebreak.

\usepackage[english]{babel} 		% Packages add functionality and style conventions to your documents. Don't edit this section!
\usepackage{fullpage}				% Eliminates wasted space
\usepackage[utf8]{inputenc}			% Necessary for character encoding
\usepackage{amsmath, amssymb,amsthm}% Required math packages
\usepackage{graphicx}				% For handling graphics
\usepackage[colorinlistoftodos]{todonotes}	% For the fancy "todo" stuff
\usepackage{hyperref}				% For clickable links in the final PDF
\usepackage{titling}				% To take less space at the top of the page with the title
\setlength{\droptitle}{-2cm}
\pretitle{\begin{flushright}\Large\scshape}
\posttitle{\par\end{flushright}}
\preauthor{\begin{flushright}\large\scshape}
\postauthor{\par\end{flushright}}
\predate{\begin{flushright}\large\scshape}
\postdate{\par\end{flushright}}


% Type `\R' for the real numbers, `\Z' for the integers, etc.

\def\Z{{\mathbb Z}}
\def\Q{{\mathbb Q}}
\def\R{{\mathbb R}}
\def\N{{\mathbb N}}

%  Here is an easy way to make lines, segments, and rays just type $ \lin{AB}$ for example.
\newcommand{\lin}[1]{\overleftrightarrow{#1}}
\newcommand{\seg}[1]{\overline{#1}}
\newcommand{\ray}[1]{\overrightarrow{#1}}

\setcounter{section}{1}

\theoremstyle{definition}
\newtheorem{theorem}{Theorem}[section]
\newtheorem{lemma}[theorem]{Lemma}
\newtheorem{prop}[theorem]{Proposition}
\newtheorem{claim}[theorem]{Claim}
\newtheorem{example}[theorem]{Example}
\newtheorem{exercise}[theorem]{Exercise}

\newtheorem*{ex}{Exercise}



\title{Math 208 HW 12}


\author{Your name goes here}  %%% Don't forget to put your name here!  %%%%


\date{Due December 5, 2020}

\begin{document}
	\maketitle
	
	%%%%%%%%%   Don't forget to use \newpage to start a new page to make 
	%%%%%%%%%   proofs/diagrams fit on the same page. 
	



\begin{ex}[10.2.7]  Prove the Translation Theorem (Theorem 10.2.8).  Namely, let $T_{AB}$ be a translation, where $A$ adn $B$ are distinct points, and let $k = \lin{AB}$.
	\begin{enumerate}
		\item If $P$ is a point on $k$, then $P' = T_{AB}(P)$ is the point on $k$ such that $P P' = A B$ and $\ray{P P'}$ is equivalent to $\ray{AB}$. IF $P$ is a point not on $k$, then $P' = T_{AB}(P)$ is on the same side of $k$ as $P$.
		\item If $n$ is any line that is perpendicular to $k$, then there exist lines $s$ and $t$ such that $T_{AB} = \rho_s \circ \rho_n = \rho_n \circ \rho_t$.
	\end{enumerate}
	
	
\end{ex}

\begin{proof} $\,$
	% % % % Type your proof here!
	\begin{enumerate}
		\item 
		\item 
	\end{enumerate}
	
\end{proof}


\vspace{1in} % leave some space in here so I can write comments





\begin{ex}[10.3.3]  Let $A$ and $B$ be two distinct points, let $\ell$ be the line that is perpendicular to $\lin{AB}$ at $A$, and let $m$ be the line that is perpendicular to $\lin{AB}$ at $B$.  Prove that $H_B \circ H_A = \rho_m \circ \rho_\ell$.  Conclude that the composition of two half-turns is a translation.
	
	
\end{ex}

\begin{proof} 
	% % % % Type your proof here!
	
\end{proof}


\vspace{1in} % leave some space in here so I can write comments








\begin{ex}[10.5.1] Give a transformational proof of the vertical angles theorem.
	
	
\end{ex}

\begin{proof} 
	% % % % Type your proof here!
	
\end{proof}


\vspace{1in} % leave some space in here so I can write comments







\end{document}
